\usepackage{xspace}
\usepackage{graphicx}% Include figure files
\usepackage{dcolumn}% Align table columns on decimal point
\usepackage{bm}% bold math
\usepackage{hyperref}% add hypertext capabilities
\usepackage{braket}
\usepackage{units}

%\usepackage{cleveref}
\usepackage{dsfont}
%\usepackage{placeins}
\usepackage{todonotes}
%\usepackage{gensymb}
\usepackage{multirow}
%\usepackage{amsmath}
%\usepackage{amsfonts}
%\usepackage{amssymb}


% 	Commands
\newcommand{\sw}{$S$-}
\newcommand{\pw}{$P$-}
\newcommand{\dw}{$D$-}
% extra
\newcommand{\babar}{BaBar}
\newcommand{\belle}{Belle\xspace}
\newcommand{\lhcb}{LHCb\xspace}
\newcommand{\aka}{{\it aka}\xspace}
\newcommand{\eg}{{\it e.g.}\xspace}
\newcommand{\ie}{{\it i.e.}\xspace}
\newcommand{\abs}[1]{\ensuremath{|#1|}}
\DeclareMathOperator{\im}{Im}
\DeclareMathOperator{\re}{Re}
\newcommand{\SigmaD}{\ensuremath{\Sigma_c^+ \bar{D}^0}\xspace}
\newcommand{\Pc}{\ensuremath{P_c(4312)^+}\xspace}
\newcommand{\jpsi}{\ensuremath{J/\psi}\xspace}
\newcommand{\jpsip}{\ensuremath{J/\psi\,p}\xspace}
%\usepackage[mathlines]{lineno}% Enable numbering of text and display math
%\linenumbers\relax % Commence numbering lines
% added  by us
\usepackage{array}   % for \newcolumntype macro
\newcolumntype{L}{>{$}l<{$}} % math-mode version of "l" column type
\newcolumntype{R}{>{$}r<{$}}
\newcolumntype{C}{>{$}c<{$}}
\usepackage{color}
\usepackage{xcolor}
\usepackage{slashed}
%\usepackage[normalem]{ulem}
%\usepackage[format=plain,justification=RaggedRight,singlelinecheck=false]{caption}
%\usepackage[format=plain,justification=centering,singlelinecheck=false]{subcaption}



% conventions
\newcommand{\nn}{\nonumber}
\newcommand{\xxx}[1]{{\color{red}\bf #1}\xspace}
% \renewcommand{\Re}{\ensuremath{\textrm{Re }}}
% \renewcommand{\Im}{\ensuremath{\textrm{Im }}}
% extra
\newcommand{\cf}{{\it cf.}\xspace}
\newcommand{\etal}{{\it et al.}\xspace}
\newcommand{\unitm}{\ensuremath{\mathds{1}}\xspace}

\newcommand{\mevnospace}{\ensuremath{{\mathrm{\,Me\kern -0.1em V}}}}
\newcommand{\gevnospace}{\ensuremath{{\mathrm{\,Ge\kern -0.1em V}}}}
\newcommand{\tevnospace}{\ensuremath{{\mathrm{\,Te\kern -0.1em V}}}}
\newcommand{\mev}{\mevnospace\xspace}
\newcommand{\gev}{\gevnospace\xspace}
\newcommand{\tev}{\tevnospace\xspace}
\newcommand{\mevp}{\ensuremath{(\!\mevnospace)}}
\newcommand{\gevp}{\ensuremath{(\!\gevnospace)}}
\newcommand{\tevp}{\ensuremath{(\!\tevnospace)}}

\newcommand{\diffd}{\ensuremath{{\mathrm{d}}}}
\newcommand{\mytitle}[1]{\vspace{.5cm}{\em #1.---}}


%%%% delete these in the final version



\usepackage{soul,color}
\definecolor{chromeyellow}{rgb}{1.0, 0.65, 0.0}
\definecolor{applegreen}{rgb}{0.55, 0.71, 0.0}
\definecolor{asparagus}{rgb}{0.53, 0.66, 0.42}
\newcommand{\adam}[1]{{\bf \color{red} Adam:  #1}\xspace}
\newcommand{\adamcor}[2]{{\color{red} Adam:\st{#1}{\bf #2}}\xspace}
\newcommand{\ale}[1]{{\bf \color{chromeyellow} Ale:  #1}\xspace}
\newcommand{\alecor}[2]{{\color{chromeyellow} Ale:\st{#1}{\bf  #2}}\xspace}
\newcommand{\vincent}[1]{{\bf \color{blue} Vincent:  #1}\xspace}
\newcommand{\vincentcor}[2]{{\color{blue} Vincent:\st{#1}{\bf  #2}}\xspace}
\newcommand{\misha}[1]{{\bf \color{applegreen} Misha:  #1}\xspace}
\newcommand{\mishacor}[2]{{\color{applegreen} Misha:\st{#1}{\bf  #2}}\xspace}
\newcommand{\cesar}[1]{{\bf \color{magenta} Cesar:  #1}\xspace}
\newcommand{\cesarcor}[2]{{\color{magenta} Cesar:\st{#1}{\bf  #2}}\xspace}
\newcommand{\miguel}[1]{{\bf \color{Cerulean} Miguel:  #1}\xspace}
\newcommand{\miguelcor}[2]{{\color{Cerulean} Miguel:\st{#1}{\bf #2}}\xspace}
\newcommand{\andrew}[1]{{\bf \color{purple} Andrew:  #1}\xspace}
\newcommand{\andrewcor}[2]{{\color{purple} Andrew:\st{#1}{\bf #2}}\xspace}
\newcommand{\arkaitz}[1]{{\bf \color{asparagus} Arkaitz:  #1}\xspace}
\newcommand{\arkaitzcor}[2]{{\color{asparagus} Arkaitz:\st{#1}{\bf #2}}\xspace}
\newcommand{\astrid}[1]{{\bf \color{Blue} Astrid:  #1}\xspace}
\newcommand{\astridcor}[2]{{\color{Blue} Astrid:\st{#1}{\bf #2}}\xspace}
